% In this file you should put the actual content of the blueprint.
% It will be used both by the web and the print version.
% It should *not* include the \begin{document}
%
% If you want to split the blueprint content into several files then
% the current file can be a simple sequence of \input. Otherwise It
% can start with a \section or \chapter for instance.
% In this file you should put the actual content of the blueprint.
% It will be used both by the web and the print version.
% It should *not* include the \begin{document}
%
% If you want to split the blueprint content into several files then
% the current file can be a simple sequence of \input. Otherwise It
% can start with a \section or \chapter for instance.

\chapter{Introduction}
This project presents a formalization of Gröbner basis theory in the Lean 4 theorem prover, establishing the mathematical infrastructure required for automated algebraic reasoning. By constructing certified implementations of fundamental algorithms and theorems, we aim to bridge the gap between computational algebra and interactive theorem proving.

\chapter{Definitions}
\begin{definition}\label{leadingTerm}
  \lean{MonomialOrder.leadingTerm}
  \leanok
  Given a nonzero polynomial \( f \in k[x] \), let
  \[
  f = c_0 x^m + c_1 x^{m-1} + \cdots + c_m,
  \]
  where \( c_i \in k \) and \( c_0 \neq 0 \) [thus, \( m = \deg(f) \)]. Then we say that \( c_0 x^m \) is the \textbf{leading term} of \( f \), written
  \[
  \operatorname{LT}(f) = c_0 x^m.
  \]
\end{definition}

\begin{definition}\label{MonomialOrder}
  \lean{MonomialOrder}
  \leanok
  \( > \) on \( k[x_1, \ldots, x_n] \) is a relation \( > \) on \( \mathbb{Z}_{\geq 0}^n \), or equivalently, a relation on the set of monomials \( x^\alpha \), \( \alpha \in \mathbb{Z}_{\geq 0}^n \), satisfying:

  \begin{enumerate}
      \item[(i)] \( > \) is a \textbf{total (or linear) ordering} on \( \mathbb{Z}_{\geq 0}^n \).

      \item[(ii)] If \( \alpha > \beta \) and \( \gamma \in \mathbb{Z}_{\geq 0}^n \), then \( \alpha + \gamma > \beta + \gamma \).

      \item[(iii)] \( > \) is a \textbf{well-ordering} on \( \mathbb{Z}_{\geq 0}^n \). This means that every nonempty subset of \( \mathbb{Z}_{\geq 0}^n \) has a smallest element under \( > \). In other words, if \( A \subseteq \mathbb{Z}_{\geq 0}^n \) is nonempty, then there is \( \alpha \in A \) such that \( \beta > \alpha \) for every \( \beta \neq \alpha \) in \( A \).
  \end{enumerate}
\end{definition}

\begin{definition}\label{IsGroebnerBasis}
  \lean{MonomialOrder.IsGroebnerBasis}
  \leanok
  \uses{leadingTerm, MonomialOrder}
  Fix a monomial order on the polynomial ring $k[x_1, \ldots, x_n]$.A finite subset $G = \{g_1, \ldots, g_t\}$ of an ideal $I \subseteq k[x_1, \ldots, x_n]$, with $I \ne \{0\}$, is said to be a \textbf{Gröbner basis} (or standard basis) if
  \[
  \langle \operatorname{LT}(g_1), \ldots, \operatorname{LT}(g_t) \rangle = \langle \operatorname{LT}(I) \rangle.
  \]
  Using the convention that $\langle \emptyset \rangle = \{0\}$, we define the empty set $\emptyset$ to be the Gröbner basis of the zero ideal $\{0\}$.
\end{definition}

\begin{definition}\label{IsRemainder}
  \lean{MonomialOrder.IsRemainder}
  \uses{leadingTerm, MonomialOrder}
  \leanok
  Fix a monomial order \(>\) on \(\mathbb{Z}_{\geq 0}^n\), and let
  \(F = (f_1, \ldots, f_s)\) be an ordered \(s\)-tuple of polynomials in \(k[x_1, \ldots, x_n]\).
  Then every \(f \in k[x_1, \ldots, x_n]\) can be written as
  \[
  f = a_1 f_1 + \cdots + a_s f_s + r,
  \]
  where \(a_i, r \in k[x_1, \ldots, x_n]\), and either \(r = 0\) or \(r\) is a linear combination, with coefficients in \(k\), of monomials, none of which is divisible by any of \(\mathrm{LT}(f_1), \ldots, \mathrm{LT}(f_s)\).
  We will call \(r\) a \textbf{remainder} of \(f\) on division by \(F\).
\end{definition}

\begin{definition}\label{S Polynomial}
  \lean{MonomialOrder.sPolynomial}
  \leanok
  The $S$-polynomial of $f$ and $g$ is the combination
  \[
  S(f, g) = \frac{x^\gamma}{\mathrm{LT}(f)} \cdot f - \frac{x^\gamma}{\mathrm{LT}(g)} \cdot g.
  \]
\end{definition}

\begin{definition}\label{Variety}
  Let \( I \subseteq k[x_1, \ldots, x_n] \) be an ideal. We will denote by \( \mathbf{V}(I) \) the set
  \[
  \mathbf{V}(I) = \{(a_1, \ldots, a_n) \in k^n \mid f(a_1, \ldots, a_n) = 0 \text{ for all } f \in I\}.
  \]
\end{definition}

\begin{definition}\label{Zerodimensional Ideal}
  \uses{Variety}
  Let \( I \subseteq k[x_1, \ldots, x_n] \) be a polynomial ideal over an algebraically closed field \( k \).
  The ideal \( I \) is called \textbf{zero-dimensional} if the associated affine variety
  \[
  \mathbf{V}(I) = \{\mathbf{a} \in k^n \mid f(\mathbf{a}) = 0 \text{ for all } f \in I\}
  \]
  is a finite set.
  \end{definition}

\chapter{Lemmas}
\section{Gröbner Basis}
\begin{lemma}\label{exists_groebner_basis}
  \lean{MonomialOrder.exists_groebner_basis}
  \uses{IsGroebnerBasis}
  Let \( I \subseteq k[x_1, \ldots, x_n] \) be an ideal. Then there exists a finite subset \( G = \{g_1, \ldots, g_t\} \) of \( I \) such that \( G \) is a Gröbner basis for \( I \).
\end{lemma}

\begin{lemma}\label{is_groebner_basis_iff}
  \lean{MonomialOrder.is_groebner_basis_iff}
  \uses{IsGroebnerBasis}
  Let \( G = \{g_1, \ldots, g_t\} \) be a finite subset of \( k[x_1, \ldots, x_n] \). Then \( G \) is a Gröbner basis for the ideal \( I = \langle G \rangle \) if and only if  for every \( f \in I \), the remainder of \( f \) on division by \( G \) is zero.
\end{lemma}


\begin{lemma}\label{groebner_basis_isRemainder_zero_iff_mem_span}
  \lean{MonomialOrder.groebner_basis_isRemainder_zero_iff_mem_span}
  \uses{IsGroebnerBasis}
  Let \( G = \{g_1, \dots, g_t\} \) be a Gröbner basis for an ideal \( I \subseteq k[x_1, \dots, x_n] \) and let \( f \in k[x_1, \dots, x_n] \). Then \( f \in I \) if and only if the remainder on division of \( f \) by \( G \) is zero.
\end{lemma}
\begin{proof}
  UTM P84
\end{proof}

\begin{lemma}\label{groebner_basis_is_basis}
  \lean{MonomialOrder.groebner_basis_is_basis}
  \uses{IsGroebnerBasis}
  Let \( G = \{g_1, \ldots, g_t\} \) be a Gröbner basis for an ideal \( I \subseteq k[x_1, \ldots, x_n] \). Then \( G \) is a basis for the vector space \( I \) over \( k \).
\end{lemma}

\begin{lemma}\label{Buchberger Criteria}
\uses{S Polynomial, groebner_basis_isRemainder_zero_iff_mem_span}
A basis \( G = \{ g_1, \ldots, g_t \} \) for an ideal \( I \) is a Gröbner basis if and only if \( S(g_i, g_j) \to_G 0 \) for all \( i \neq j \).
\end{lemma}
\begin{proof}
 UTM P86
\end{proof}

\section{Zero-dimensional Ideal}\label{Zero-dimensional Ideal}
\begin{lemma}\label{fg_span_iff_fg_span_finset_subset}
  \lean{Ideal.fg_span_iff_fg_span_finset_subset}
  Let \( I \subseteq k[x_1, \ldots, x_n] \) be an ideal. Then \( I \) is finitely generated if and only if there exists a finite subset \( F \subseteq I \) such that \( I = \langle F \rangle \).
\end{lemma}

\begin{lemma}\label{leadingTerm_ideal_span_monomial}
  \lean{MonomialOrder.leadingTerm_ideal_span_monomial'}
  \leanok
  \uses{leadingTerm, MonomialOrder}
  \[
  \langle \operatorname{lt}(G) \rangle = \left\langle \{ x^t : t \in \{ \operatorname{multideg}(p) : p \in G \setminus \{0\} \} \right\rangle
  \]
\end{lemma}

\begin{lemma}\label{remainder_mem_ideal_iff}
  \lean{MonomialOrder.remainder_mem_ideal_iff}
  \uses{IsRemainder, IsGroebnerBasis}
  Let \( I \subseteq k[x_i : i \in \sigma] \) be an ideal, and let \( G \subseteq I \) be a finite subset.
  If \( r \) is a generalized remainder of \( p \) upon division by \( G \),
  then \( r \in I \) if and only if \( p \in I \)
\end{lemma}

\begin{lemma}\label{remainder_sub_remainder_mem_ideal}
  \lean{MonomialOrder.remainder_sub_remainder_mem_ideal}
  \uses{IsRemainder, IsGroebnerBasis}
  Let \( I \subseteq k[x_i : i \in \sigma] \) be an ideal, and let \( G \subseteq I \) be a finite subset.
  Suppose \( r_1 \) and \( r_2 \) are generalized remainders of a polynomial \( p \) upon division by \( G \).
  Then \( r_1 - r_2 \in I \).
\end{lemma}

\begin{lemma}\label{verify_zero_dim_ideal}
  \uses{Zero-dimensional Ideal}
  Let \( I \subseteq k[x_1, \ldots, x_n] \) be an ideal and fix a monomial ordering on \( k[x_1, \ldots, x_n] \). For each \( i \), \( 1 \leq i \leq n \), there exists some \( m_i \geq 0 \) such that \( x_i^{m_i} \in \langle \mathrm{LT}(I)\rangle \). Then I is Zero-dimensional Ideal.
\end{lemma}

\begin{lemma}\label{V_I_equal_V_F}
  \uses{Variety}
  \( \mathbf{V}(I) \) is an affine variety. In particular, if \( I = \langle f_1, \ldots, f_s \rangle \), then
  \[
  \mathbf{V}(I) = \mathbf{V}(f_1, \ldots, f_s).
  \]
\end{lemma}
\begin{proof}
  By definition
\end{proof}

\begin{lemma}\label{V_G_equal_V_I}
  \uses{Variety, V_I_equal_V_F, Zero-dimensional Ideal}
  Let $I \subseteq k[x_1,\ldots,x_n]$ be a zero-dimensional ideal and $G = \{g_1,\ldots,g_t\}$ a Gröbner Basis for $I$. Then the affine variety defined by $G$ equals the variety of $I$:
  \[
  \mathbf{V}(G) = \mathbf{V}(I)
  \]
\end{lemma}
