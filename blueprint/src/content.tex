% In this file you should put the actual content of the blueprint.
% It will be used both by the web and the print version.
% It should *not* include the \begin{document}
%
% If you want to split the blueprint content into several files then
% the current file can be a simple sequence of \input. Otherwise It
% can start with a \section or \chapter for instance.
% In this file you should put the actual content of the blueprint.
% It will be used both by the web and the print version.
% It should *not* include the \begin{document}
%
% If you want to split the blueprint content into several files then
% the current file can be a simple sequence of \input. Otherwise It
% can start with a \section or \chapter for instance.

\chapter{Introduction}
This project presents a formalization of Gröbner basis theory in the Lean 4 theorem prover, establishing the mathematical infrastructure required for automated algebraic reasoning. By constructing certified implementations of fundamental algorithms and theorems, we aim to bridge the gap between computational algebra and interactive theorem proving.

\chapter{Definitions}
\begin{definition}\label{Leading Term}
  \leanok
  Given a nonzero polynomial \( f \in k[x] \), let
  \[
  f = c_0 x^m + c_1 x^{m-1} + \cdots + c_m,
  \]
  where \( c_i \in k \) and \( c_0 \neq 0 \) [thus, \( m = \deg(f) \)]. Then we say that \( c_0 x^m \) is the \textbf{leading term} of \( f \), written
  \[
  \operatorname{LT}(f) = c_0 x^m.
  \]
\end{definition}

\begin{definition}\label{Gröbner Basis}
  \leanok
  Fix a monomial order on the polynomial ring $k[x_1, \ldots, x_n]$.A finite subset $G = \{g_1, \ldots, g_t\}$ of an ideal $I \subseteq k[x_1, \ldots, x_n]$, with $I \ne \{0\}$, is said to be a **Gröbner basis** (or **standard basis**) if
  \[
  \langle \operatorname{LT}(g_1), \ldots, \operatorname{LT}(g_t) \rangle = \langle \operatorname{LT}(I) \rangle.
  \]
  Using the convention that $\langle \emptyset \rangle = \{0\}$, we define the empty set $\emptyset$ to be the Gröbner basis of the zero ideal $\{0\}$.
\end{definition}

\begin{definition}\label{S Polynomial}

\end{definition}


\chapter{Lemmas}

